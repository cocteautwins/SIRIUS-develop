\hypertarget{basis_basis1}{}\section{L\+A\+P\+W+lo basis}\label{basis_basis1}
L\+A\+P\+W+lo basis consists of two different sets of functions\+: L\+A\+P\+W functions $ \varphi_{{\bf G+k}} $ defined over entire unit cell\+: \[ \varphi_{{\bf G+k}}({\bf r}) = \left\{ \begin{array}{ll} \displaystyle \sum_{L} \sum_{\nu=1}^{O_{\ell}^{\alpha}} a_{L\nu}^{\alpha}({\bf G+k})u_{\ell \nu}^{\alpha}(r) Y_{\ell m}(\hat {\bf r}) & {\bf r} \in {\rm MT} \alpha \\ \displaystyle \frac{1}{\sqrt \Omega} e^{i({\bf G+k}){\bf r}} & {\bf r} \in {\rm I} \end{array} \right. \] and Bloch sums of local orbitals defined inside muffin-\/tin spheres only\+: \[ \begin{array}{ll} \displaystyle \varphi_{j{\bf k}}({\bf r})=\sum_{{\bf T}} e^{i{\bf kT}} \varphi_{j}({\bf r - T}) & {\rm {\bf r} \in MT} \end{array} \] Each local orbital is composed of radial and angular parts\+: \[ \varphi_{j}({\bf r}) = \phi_{\ell_j}^{\zeta_j,\alpha_j}(r) Y_{\ell_j m_j}(\hat {\bf r}) \] Radial part of local orbital is defined as a linear combination of radial functions (minimum two radial functions are required) such that local orbital vanishes at the sphere boundary\+: \[ \phi_{\ell}^{\zeta, \alpha}(r) = \sum_{p}\gamma_{p}^{\zeta,\alpha} u_{\ell \nu_p}^{\alpha}(r) \]

Arbitrary number of local orbitals can be introduced for each angular quantum number (this is highlighted by the index $ \zeta $).

Radial functions are m-\/th order (with zero-\/order being a function itself) energy derivatives of the radial Schrödinger equation\+: \[ u_{\ell \nu}^{\alpha}(r) = \frac{\partial^{m_{\nu}}}{\partial^{m_{\nu}}E}u_{\ell}^{\alpha}(r,E)\Big|_{E=E_{\nu}} \] 
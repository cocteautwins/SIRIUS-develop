\hypertarget{sym_section1}{}\section{Preliminary notes}\label{sym_section1}
S\+I\+R\+I\+U\+S uses Spglib to find the spacial symmetry operations. Spglib defines symmetry operation in fractional coordinates\+: \[ {\bf x'} = \{ {\bf R} | {\bf t} \} {\bf x} \equiv {\bf R}{\bf x} + {\bf t} \] where {\bfseries R} is the proper or improper rotation matrix with elements equal to -\/1,0,1 and determinant of 1 (pure rotation) or -\/1 (rotoreflection) and {\bfseries t} is the fractional translation, associated with the symmetry operation. The inverse of the symmetry operation is\+: \[ {\bf x} = \{ {\bf R} | {\bf t} \}^{-1} {\bf x'} = {\bf R}^{-1} ({\bf x'} - {\bf t}) = {\bf R}^{-1} {\bf x'} - {\bf R}^{-1} {\bf t} \]

We will always use an {\itshape active} transformation (transformation of vectors or functions) and never a passive transformation (transformation of coordinate system). However one should remember definition of the function transformation\+: \[ \hat {\bf P} f({\bf r}) \equiv f(\hat {\bf P}^{-1} {\bf r}) \]

It is straightforward to get the rotation matrix in Cartesian coordinates. We know how the vector in Cartesian coordinates is obtained from the vector in fractional coordinates\+: \[ {\bf v} = {\bf L} {\bf x} \] where {\bfseries L} is the 3x3 matrix which clomuns are three lattice vectors. The backward transformation is simply \[ {\bf x} = {\bf L}^{-1} {\bf v} \] Now we write rotation operation in fractional coordinates and apply the backward transformation to Cartesian coordinates\+: \[ {\bf x'} = {\bf R}{\bf x} \rightarrow {\bf L}^{-1} {\bf v'} = {\bf R} {\bf L}^{-1} {\bf v} \] from which we derive the rotation operation in Cartesian coordinates\+: \[ {\bf v'} = {\bf L} {\bf R} {\bf L}^{-1} {\bf v} \] 